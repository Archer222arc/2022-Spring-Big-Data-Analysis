\usepackage{amsmath, amsfonts, amsthm}
\usepackage{nicematrix}

%\usepackage{graphicx, epstopdf}
\usepackage{color}
\usepackage{cite}
\usepackage{indentfirst}
\usepackage{geometry, graphicx}
\usepackage[title]{appendix}
\usepackage{algorithm, algorithmic}
\usepackage{bm}
\usepackage[hidelinks]{hyperref}
\usepackage{multirow}
\usepackage[tight]{subfigure}
%\usepackage{ulem}
\geometry{left = 5em, right = 5em}
\usepackage{listings}
\usepackage{xcolor}
\usepackage{subfigure} 
%% notation macro
\newcommand{\OB}{\Ocal\B}
\newcommand{\Rt}{\operatorname{Rt}}
\newcommand{\Grad}{\operatorname{grad}}
\newcommand{\sign}{\operatorname{sign}}
\newcommand{\diag}{\operatorname{diag}}
\newcommand{\tfo}{\textbf{1}}
\newcommand{\Ccal}{\mathcal C}
\newcommand{\Cbb}{\mathbb C}
\newcommand{\Cbf}{\mathbf C}
\newcommand{\E}{\mathcal E}
\newcommand{\F}{\mathcal F}
\newcommand{\T}{\mathcal T}
\newcommand{\I}{\mathcal I}
\newcommand{\Ocal}{\mathcal O}
\newcommand{\Ibf}{\mathbf I}
\newcommand{\Y}{\mathcal Y}
\newcommand{\Ybf}{\mathbf Y}
% \newcommand{\U}{\mathcal U}
\newcommand{\R}{\mathbb R}
\renewcommand{\P}{\mathcal P}
\newcommand{\uP}{ \mathcal \uline P}
\newcommand{\V}{\mathcal V}
\newcommand{\A}{\mathcal A}
\newcommand{\B}{\mathcal B}
\newcommand{\G}{\mathcal G}
\newcommand{\Hcal}{\mathcal Hcal}
%\newcommand{\R}{\mathbb R^2}
\newcommand{\Z}{\mathbb Z}
\newcommand{\Eb}{\mathbb E}
% \newcommand{\C}{\mathbb C}
\newcommand{\laplacian}{\triangle}
\newcommand{\grad}{\nabla}
\renewcommand{\div}{\textrm{div~}}
% cf
\newcommand{\iprod}[2]{\left\langle #1, #2 \right\rangle}
\newcommand{\II}[1]{\mathbb{1}\left\{#1\right\}}
\newcommand{\nrm}[1]{\left\|#1\right\|}
\newcommand{\bH}{\mathbf{H}}
\newcommand{\eps}{\varepsilon}
\newcommand{\DD}{\mathbb{D}}
\newcommand{\RR}{\mathbb{R}}
\newcommand{\cO}{\mathcal{O}}
\newcommand{\cB}{\mathcal{B}}
\newcommand{\PP}{\mathbb{P}}
\newcommand{\EE}{\mathbb{E}}
\newcommand{\relu}{\operatorname{ReLU}}
\renewcommand{\II}[1]{\mathbb{I}\left\{#1\right\}\}}

\newcommand{\diff}[2]{\frac{\partial #1}{\partial #2}}
\newcommand{\difff}[3]{\frac{\parial #1^2}{\partial #2 \partial #3}}
\newcommand{\diFF}[2]{\frac{\partial #1^2}{\partial^2 #2}}
\newcommand{\diam}{\text{ diam }}
%% non-noation macro
\newcommand{\IN}{\text{  in  }}
\newcommand{\ON}{\text{  on  }}
\newcommand{\st}{\text{s.t.  }}
\newcommand{\tbc}{{\color{red}[TBC]}}
\newcommand\ldq\textquotedblleft
\newcommand\rdq\textquotedblright{}
\newcommand\mb\mathbb
\newcommand\mf\mathbf
\newcommand\tf\textbf
\newcommand{\revise}[1]{{\color{blue}#1}}

\newcommand{\return}{\textbf{return~}}
\DeclareMathOperator{\argmin}{arg~min}
\DeclareMathOperator{\argmax}{arg~max}
%% enviorment
\newtheorem{proposition}{Proposition}
\newtheorem{definition}{Definition}
\newtheorem{corollary}{Corollary}
\newtheorem{remark}{Remark}
\newtheorem{assumption}{Assumption}
\newtheorem{Proposition}{Proposition}
\newtheorem{Definition}{Definition}
\newtheorem{Corollary}{Corollary}
\newtheorem{Remark}{Remark}
\newtheorem{Assumption}{Assumption}
\newtheorem{Condition}{Condition}
\newtheorem{condition}{Condition}
\newtheorem{theorem}{Theorem}
\newtheorem{lemma}{Lemma}
\setlength{\parindent}{1.5em}
\definecolor{mygreen}{rgb}{0,0.6,0}
\definecolor{mygray}{rgb}{0.5,0.5,0.5}
\definecolor{mymauve}{rgb}{0.58,0,0.82}
\lstset{ %
	backgroundcolor=\color{white},      % choose the background color
	basicstyle=\footnotesize\ttfamily,  % size of fonts used for the code
	columns=fullflexible,
	tabsize=4,
	breaklines=true,               % automatic line breaking only at whitespace
	captionpos=b,                  % sets the caption-position to bottom
	commentstyle=\color{green},  % comment style
	escapeinside={\%*}{*)},        % if you want to add LaTeX within your code
	keywordstyle=\color{blue},     % keyword style
	stringstyle=\color{mymauve}\ttfamily,  % string literal style
	frame=single,
	rulesepcolor=\color{red!20!green!20!blue!20},
	% identifierstyle=\color{red},
	language=matlab,
	numbers=left,
}
\title{\textbf{\showtitle}}
\author{\showauthor}
\usepackage{indentfirst}
\usepackage{fancyhdr}  
\pagestyle{fancy}

\renewcommand{\algorithmicrequire}{\textbf{Input}}
\renewcommand{\algorithmicensure}{\textbf{Output}}
%\lhead{\textbf {\showtopic} }
%\chead{} 
%\rhead{\textbf {\showabs} }
%\lfoot{} 
%\cfoot{\thepage}
%\rfoot{} 
%\renewcommand{\headrulewidth}{0.4pt} 